\documentclass[12pt, letterpaper]{article}
\usepackage{listings}
\usepackage{color}
% \usepackage{placeins}
 
\definecolor{codegreen}{rgb}{0,0.6,0}
\definecolor{codegray}{rgb}{0.5,0.5,0.5}
\definecolor{codepurple}{rgb}{0.58,0,0.82}
\definecolor{backcolour}{rgb}{0.95,0.95,0.92}
 
\lstdefinestyle{mystyle}{
    backgroundcolor=\color{backcolour},   
    commentstyle=\color{codegreen},
    keywordstyle=\color{magenta},
    numberstyle=\tiny\color{codegray},
    stringstyle=\color{codepurple},
    basicstyle=\footnotesize,
    breakatwhitespace=true,         
    breaklines=true,                 
    captionpos=b,                    
    keepspaces=true,                 
    numbers=left,                    
    numbersep=5pt,                  
    showspaces=false,                
    showstringspaces=false,
    showtabs=false,                  
    tabsize=2
}
 
\lstset{style=mystyle}

\title{Social Media Tracking Documentation}
\author{Cedric Tan - Roar Sports}
\date{April 2019}

\begin{document}
\maketitle
\abstract{The point of this documentation is to outline the details of the Python program that will track the social media following of Click Agency clients. The actual data is transported into an Excel file where independent analysis is conducted.}

\newpage

\tableofcontents
\newpage

\section{Introduction}
This document aims to spell out the documentation of the Python programme that will automate social media tracking for Roar Sports. The program focuses on the key social media platforms for scraping follower data. Further analysis will be done with the excel content at a later stage. The following social media platforms are included in this documentation:
\begin{enumerate}
	\item Facebook
	\item Twitter
	\item Twitch
	\item Youtube
	\item Instagram
\end{enumerate}

Following this guide will help you understand how the program works.

\section{Facebook}
Beginning with the Facebook interface, the program will request access to the Facebook \textbf{Graph API} by authenticating with the website, making a request to access for the data of a particular page or profile and subsequently saving this profile data to a database where we can print it all out on excel.



\section{Twitter}
With Twitter, the method will utilise \textbf{Tweepy} which is a library that allows for quick access to the Twitter API.

\subsection{Code Rundown}
First we will begin with the dependencies required to gain the data:

\begin{lstlisting}[language=python]
from tweepy import OAuthHandler
from tweepy import API
from tweepy import Cursor
import sys
\end{lstlisting}

Here is the rundown on each dependency:
\begin{itemize}
	\item Tweepy is a Python library that we can use to access the Twitter API
	\item OAuthHandler will allow us to painlessly authorise access to get data from the API itself
	\item API is the main access point for Tweepy to get the data
	\item Cursor allows for pagination and access to large volumes of data that we might need to segment into sections i.e. pages
	\item Sys provides some basic functionality for the interpreter
\end{itemize}

Then we will set up the keys to authorise our access to the Twitter API:

\begin{lstlisting}[language=python]

# These are the keys from the app created on the developer account

consumer_key = "..." 
consumer_secret = "..."
access_token = "..."
access_secret_token = "..."

# This begins the authorisation process with the Twitter API so that the program can authenticate it self with the website

auth = OAuthHandler(consumer_key, consumer_secret)
auth.set_access_token(access_token, access_secret_token)
auth_api = API(auth)
\end{lstlisting}

Then with all parts authorised, we can begin to get the data from the API itself through simple get commands.

\begin{lstlisting}[language=python]

# We will first get the Twitter handle of the account that we want to check out:

user = auth_api.get.user("barackobama")

# Having gotten the object with the get.user function, in this case Barack Obama's account, we can execute some functions on it like:

screen_name = user.screen_name # Take the user's screen name
follower_count = user.followers_count # Take the user's follower count

# Then printing these, we return values specific to the account

print (screen_name)
print (follower_count)

# This returns:
	BarackObama
	105699616 # 17/04/2019 12:07pm

\end{lstlisting}

\subsection{Data Retrieval}
From the previous section, we can just iterate the code over and over again from our database of Twitter Handles. This can either be hard coded into the system or taken from an Excel and imported in.

The iteration of code will be a simple for loop such that the output will show the each line of data specific to the Twitter handle. Taking the code from above, we can implement the loop below:

\begin{lstlisting}[language=python]
# Defining a dictionary
name_dictionary = ["name1", "name2", "name3", "name4"]

for name in name_dictionary:
	user = auth_api.get.user(name)
	screen_name = user.screen_name
	follower_count = user.followers_count
	
	print(screen.name)
	print(follower.count)
	
# This will return
	name1
	name1_follower_count
	...
	name4
	name4_follower_count

\end{lstlisting}

\end{document}